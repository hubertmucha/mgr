% \chapter{Wprowadzenie}
% \label{cha:wprowadzenie}

% \LaTeX~jest systemem składu umożliwiającym tworzenie dowolnego typu dokumentów (w~szczególności naukowych i technicznych) o wysokiej jakości typograficznej (\cite{Dil00}, \cite{Lam92}). Wysoka jakość składu jest niezależna od rozmiaru dokumentu -- zaczynając od krótkich listów do bardzo grubych książek. \LaTeX~automatyzuje wiele prac związanych ze składaniem dokumentów np.: referencje, cytowania, generowanie spisów (treśli, rysunków, symboli itp.) itd.

% \LaTeX~jest zestawem instrukcji umożliwiających autorom skład i wydruk ich prac na najwyższym poziomie typograficznym. Do formatowania dokumentu \LaTeX~stosuje \TeX a (wymiawamy 'tech' -- greckie litery $\tau$, $\epsilon$, $\chi$). Korzystając z~systemu składu \LaTeX~mamy za zadanie przygotować jedynie tekst źródłowy, cały ciężar składania, formatowania dokumentu przejmuje na siebie system.

% %---------------------------------------------------------------------------

% \section{Cele pracy}
% \label{sec:celePracy}


% Celem poniższej pracy jest zapoznanie studentów z systemem \LaTeX~w zakresie umożliwiającym im samodzielne, profesjonalne złożenie pracy dyplomowej w systemie \LaTeX.

% \subsection{Jakiś tytuł}

% \subsubsection{Jakiś tytuł w subsubsection}


% \subsection{Jakiś tytuł 2}

% %---------------------------------------------------------------------------

% \section{Zawartość pracy}
% \label{sec:zawartoscPracy}

% W rodziale~\ref{cha:pierwszyDokument} przedstawiono podstawowe informacje dotyczące struktury dokumentów w \LaTeX u. Alvis~\cite{Alvis2011} jest językiem 


\chapter{Część teoretyczna}
U podsataw zarówno uwczesnej jak i obecnej elektorniki leży potrzeba przetwarzania sygnałów. To w sygnałach zawarta jest informacja której
przetwarzanie jest głownym i jedynym celem konstułowanych układów. Potrzeba ta wynikia z obecnego spectrum w którym ów ukłądu znajują zastosowanie.
Brażne takie jak branża medyczna, w której natychmiastowo nasuwającym się zastosowaniem układów elektonicznych w celu translacji sygnałow
pochodzących prosto z ludzkiego ciała, wymaga przetwarzania ów sygnału.
Branża automotive, w której sygnały odczytywane z różnorakich czujników rówież wymaga przetważania sygnału.

Sygnał w postaci fali elektomagnetycznej przenoszonej przez odpowiednie miedium może być wyrażony i określony na wiele sposóbów.
Autor pracy na potrzeby analizowanego zagadnienia tj. Projekt układu próbkująco-pamiętającego na potrzeby szybkich przetworników ADC w technologii submikronowej,
skupia swoją uwagę na sygnale w formie analogowej.

\section{Sygnał analogowy, próbkowanie sygnału}
\subsubsection{Sygnał analogowy}

Każdy wystepujacy w naturze sygnał jest sygnałem analogowym. Sygnał analogowy to ciągły w czasie i wartości ciąg wartości, ktory dla każdej chwili czasowej obserwacji tego sygnału może
przybierać każdą możliwą wartość ze zbioru. Zbiór określajacy potenacjalne wartosći tego sygnału może być zarówno ograniczony jak i nieograniczony.
Przykładowymi zródłami sygnaółów w postaci analogowej są: sygnał mowy ludzkiej, promieniowanie elektomegnetyczne osberwowane przez teleskop, impuls elektryczny 
obserwowany w nerwie w ludzkim ciele. 

\subsubsection{Próbkowanie sygnału}
Próbkowanie sygnału jest jedna z postwowych operacji szeroko stosowanych w torach przetwarzania sygnałów.
Dyskretyzacja to proces tworzenia sygnału dyskretnego, reprezentującego sygnał ciągły za pomocą ciągu wartości nazywanych próbkami.

Twierdzenie o próbkowaniu, (twierdzenie Nyquista–Shannona) wyznacza minimalna deteminuje minimalną częstotliwość, która pozwala na
wierne odtworzenie sygnału $x(t)$, z sygnału dyskretnego $x^*(t)$ złożonego z próbek dnaego sygnału ciągłego $x(t)$.

Aby zachować pewność jakości odzworowania sygnału próbkowanego widmo sygnału oryginalnego $f_o$ przesunięte o możliwie wszystkie całkowite wieloktonośći
częstotliwośći próbkowania $f_{s}$ nie nachodzą na siebie. Praktyczne zastowanie twierdzenia o próbowaniu wymaga spełnienia rówanania (\ref{eq:probkowanie}).
\begin{equation}
    f_s > 2f_o
    \label{eq:probkowanie}
\end{equation}



\section{Zawartość pracy}














